\documentclass[a4paper]{article}

\usepackage{listings}

\title{Example: Calculate the gravity gradient tensor from a DEM file}
\author{Leonardo Uieda}

\begin{document}

\maketitle

\lstset{numbers=left,
        basicstyle=\footnotesize,
        breaklines=true}

This document intends to demonstrate how to calculate the gravity gradient
tensor (GGT) due to topographic masses. To do that we need:
\begin{enumerate}
    \item A DEM file with lon, lat, and height information;
    \item Convert the DEM information into a tesseroid model;
    \item Assign correct densities to the tesseroids (in the oceans it should be different);
    \item Calculate the 6 components of the GGT;
    \item Make nice color plots of the results (we'll be using Python for this);
\end{enumerate}



\section{The DEM file}




\lstinputlisting[language=Python]{plot_dem.py}



\end{document}